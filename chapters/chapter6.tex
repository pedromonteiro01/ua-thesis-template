\chapter{Conclusions and Future Work}
\label{chapter:conclusion}

\newenvironment{conclusion}
{\quote\itshape}
{\endquote}

\begin{conclusion}
\end{conclusion}

\section{Summary}
This thesis started with an extensive literature review in the domain of object detection, focusing on 
identifying the most prevalent algorithms, architectures, datasets, and evaluation metrics. Following this 
research, the next phase involved the creation of a comprehensive dataset specialized for object 
detection tasks. Subsequently, an object detection algorithm was trained and tested using this dataset.

Before the implementation, a design phase was conducted to outline the main components necessary for the application.
The project then progressed to the implementation phase, where an application was developed, integrating all 
modules. Attention was given to scalability and performance optimizations to ensure the application 
could handle real-world demands efficiently.

In the final stage, a demonstration was conducted with Coimbra police committee to gather feedback and assess 
the application's effectiveness. The feedback from this evaluation was crucial in determining the application's 
practical viability and potential areas for improvement.

\section{Conclusions}
This work aimed to create a real-time web application for detecting weapons and triggering alerts.
The trained model achieved an impressive accuracy of 0.973, coupled with high precision and recall. This ensured 
reliable and consistent performance in real-world scenarios.

The integration of this pretrained model into a web application has significant practical implications. The application 
consumes data in real time, enabling it to detect and trigger alerts immediately. Each detected frame is saved for 
further analysis, which is crucial for monitoring and responding to events as they unfold.

In real-world settings, this application can be used to monitor security cameras. The ability to process data in 
real time and trigger alerts can significantly influence practices in these fields by improving response times and 
enabling proactive measures. For instance, the system can alert personnel to potential 
threats instantly, allowing for rapid intervention.

Despite its successes, this study has limitations. While the model's performance is high, it encounters difficulties in 
specific scenarios. One significant limitation is the detection of weapons when they are partially concealed, such as in 
an attacker's pocket. The model's ability to identify threats is also challenged when the weapon is on a dark background, 
where contrast and visibility are reduced. Although the dataset includes images of these challenging scenarios, the model 
can still fail under these conditions. Additionally, the real-time processing capability, although robust, may face 
challenges with extremely high data throughput or in environments with limited computational resources. 

\section{Future Work}
The current implementation of this surveillance system provides a robust platform for monitoring through multiple 
camera feeds. However, to enhance its utility and performance, several improvements are proposed for future iterations.

\begin{itemize}
    \item Implementation of Auto-Scaling Mechanisms: allow the system to 
    dynamically adjust its resources based on workload demands. This will involve integrating monitoring tools 
    and automated scaling policies.
    Additionally, exploring container orchestration platforms 
    like Kubernetes will be essential for managing resource allocation and scaling efficiently.
    \item Upgrading System Hardware: the addition of real-time object detection across multiple feeds needs a 
    corresponding upgrade in system hardware. As the number of cameras and the computational load increase, there is 
    a substantial impact on system performance. To mitigate this, an enhancement in both CPU and \ac{gpu} capabilities will 
    be required. This upgrade will ensure that the system can handle the increased processing demands without degradation 
    in performance;
    \item Exploration of Advanced Object Detection Models: the implementation of newer versions of the \ac{yolo} 
    algorithm should be explored. These versions can offer improvements in detection accuracy;
    \item Dataset Adaptation: change the dataset according to Coimbra police security camera footages and settings. This
     will involve customizing the dataset to reflect the specific environments and conditions encountered in Coimbra. 
\end{itemize}